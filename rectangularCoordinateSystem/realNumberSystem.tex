\documentclass{ximera}

\input{../preamble}
\newcommand{\IC}{\mathbb{C}}
\newcommand{\ID}{\mathbb{D}}
\newcommand{\IF}{\mathbb{F}}
\newcommand{\IM}{I\!\! M}
\newcommand{\IN}{\mathbb{N}}
\newcommand{\IP}{\mathbb{P}}
\newcommand{\IQ}{\mathbb{Q}}
\newcommand{\IR}{\mathbb{R}}
\newcommand{\IS}{\mathbb{S}}
\newcommand{\IT}{\mathbb{T}}
\newcommand{\IZ}{\mathbb{Z}}
\newcommand{\FC}{\frak{C}}
\newcommand{\eps}{{\varepsilon}}
\newcommand{\CA}{\mathcal{A}}
\newcommand{\CB}{\mathcal{B}}
\newcommand{\CC}{\mathcal{C}}
\newcommand{\CD}{\mathcal{D}}
\newcommand{\CE}{\mathcal{E}}
\newcommand{\CF}{\mathcal{F}}
\newcommand{\CL}{\mathcal{L}}
\newcommand{\CM}{\mathcal{M}}
\newcommand{\CN}{\mathcal{N}}
\newcommand{\CP}{\mathcal{P}}
\newcommand{\CQ}{\mathcal{Q}}
\newcommand{\CS}{\mathcal{S}}
\newcommand{\CT}{\mathcal{T}}
\newcommand{\CR}{\mathcal{R}}
\newcommand{\CK}{\mathcal{K}}
\newcommand{\CU}{\mathcal{U}}
\newcommand{\CX}{\mathcal{X}}

\outcome{}

\title{The $\IR$eal Number System}

\author{Menassie Ephrem}

\begin{document}

\begin{abstract}
We learn the real number system here.
\end{abstract}

\maketitle
Before we look at the rectangular coordinate system we will briefly review the real number system.

% let's start a new thought -- a new section
{The set of natural numbers,} denoted by $\IN$, is the set $$\IN = \{1,~2,~3, ~\ldots\}.$$
Although this is the most accepted convention, some literature include $0$ in this set of natural numbers.

{The set of integers,} denoted by $\IZ$, is the set $$\IZ = \{\ldots, ~-3, ~-2, ~-1, ~0, ~1,~2, ~3, ~\ldots\}.$$  

{The set of rational numbers.} The next set of numbers, $\IQ$ is the set of all `valid' ratios of integers.  These are fractions of integers where the denominator is not allowed to be zero.  In short $$\IQ = \{ a/b~ |~ a \in \IZ,~ b \in \IZ, \text{ and } b \neq 0\}.$$

It is not hard to see that $\IZ$ is a subset of $\IQ$. We will use $\subseteq$ to denote subset.  In this care $\IZ \subseteq \IQ$. It is interesting to note that even though $\IN$ is clearly a proper subset of $\IQ$, these two sets have the same number of elements.



\end{document}